\documentclass[12pt,letterpaper,titlepage]{report}
\usepackage{algorithm2e}
\usepackage{listings}
\usepackage{geometry}
\usepackage{graphicx}
\usepackage[hypertexnames=false, pdftex]{hyperref}
%%%%%%%%%%%%%%%%%%%%%%%%%%%%%%%%%%%%%%%%%%%%%%%%%%%%%%%%%%%%%%%%%%%%
% use:$ pdflatex ParallelDetails.tex
%%%%%%%%%%%%%%%%%%%%%%%%%%%%%%%%%%%%%%%%%%%%%%%%%%%%%%%%%%%%%%%%%%%%
\author{Berenger Bramas}
\title{ScalFmm - Parallel Algorithms (Draft)}
\date{August 11, 2011}

%% Package config
\lstset{language=c++, frame=lines}
\restylealgo{boxed}
\geometry{scale=0.8, nohead}
\hypersetup{ colorlinks = true, linkcolor = black, urlcolor = blue, citecolor = blue }
%% Remove introduction numbering
\setcounter{secnumdepth}{-1}
%%%%%%%%%%%%%%%%%%%%%%%%%%%%%%%%%%%%%%%%%%%%%%%%%%%%%%%%%%%%%%%%%%%%
%%%%%%%%%%%%%%%%%%%%%%%%%%%%%%%%%%%%%%%%%%%%%%%%%%%%%%%%%%%%%%%%%%%%
\begin{document}
\maketitle{}
\newpage
\tableofcontents
\newpage
%%%%%%%%%%%%%%%%%%%%%%%%%%%%%%%%%%%%%%%%%%%%%%%%%%%%%%%%%%%%%%%%%%%%
%%%%%%%%%%%%%%%%%%%%%%%%%%%%%%%%%%%%%%%%%%%%%%%%%%%%%%%%%%%%%%%%%%%%
\section{Introduction}
In this document we introduce the principles and the algorithms used in our library to run in a distributed environment using MPI.
The algorithms in this document may not be up to date comparing to those used in the code.
We advise to check the version of this document and the code to have the latest available.
%%%%%%%%%%%%%%%%%%%%%%%%%%%%%%%%%%%%%%%%%%%%%%%%%%%%%%%%%%%%%%%%%%%%
%%%%%%%%%%%%%%%%%%%%%%%%%%%%%%%%%%%%%%%%%%%%%%%%%%%%%%%%%%%%%%%%%%%%
\chapter{Building the tree in Parallel}
\section{Description}
The main motivation to create a distributed version of the FMM is to run large simulations.
These ones contain more particles than a computer can host which involves using several computers.
Moreover, it is not reasonable to ask a master process to load an entire file and to dispatch the data to others processes. Without being able to know the entire tree it may send randomly the data to the slaves.
To override this situation, our solution can be viewed as a two steps process.
First, each node loads a part of the file to possess several particles.
After this task, each node can compute the Morton index for the particles he had loaded.
The Morton index of a particle depends of the system properties but also of the tree height.
If we want to choose the tree height and the number of nodes at run time then we cannot pre-process the file.
The second step is a parallel sort based on the Morton index between all nodes with a balancing operation at the end.
%%%%%%%%%%%%%%%%%%%%%%%%%%%%%%%%%%%%%%%%%%%%%%%%%%%%%%%%%%%%%%%%%%%%
%%%%%%%%%%%%%%%%%%%%%%%%%%%%%%%%%%%%%%%%%%%%%%%%%%%%%%%%%%%%%%%%%%%%
\section{Load a file in parallel}
We use the MPI $I/O$ functions to split a file between all the mpi processes.
The prerequisite to make the splitting easier is to have a binary file.
Thereby, using a very basic formula each node knows which part of the file it needs to load.
\begin{equation}
size per proc \leftarrow \left (file size - header size \right ) / nbprocs
\end{equation}
\begin{equation}
offset \leftarrow header size + size per proc .\left ( rank - 1 \right )
\end{equation}
\newline
We do not use the view system to read that data as it is used to write. The MPI\_File\_read is called as described in the fallowing $C++$ code.
\begin{lstlisting}
// From FMpiFmaLoader
MPI_File_read_at(file, headDataOffSet + startPart * 4 * sizeof(FReal),
                 particles, int(bufsize), MPI_FLOAT, &status);
\end{lstlisting}
Our files are composed by a header fallowing by all the particles.
The header enables to check several properties as the precision of the file.
Finally, a particle is represented by four decimal values: a position and a physical value.
\newline
\underline{Remark:} The MPI IO function do not work if we use a MPI\_Initthread(MPI\_THREAD\_MULTIPLE) and a version above 1.5.1.

%%%%%%%%%%%%%%%%%%%%%%%%%%%%%%%%%%%%%%%%%%%%%%%%%%%%%%%%%%%%%%%%%%%%
%%%%%%%%%%%%%%%%%%%%%%%%%%%%%%%%%%%%%%%%%%%%%%%%%%%%%%%%%%%%%%%%%%%%
\section{Sorting the particles}
Once each node has a set of particles we need to sort them.
This problem boils down to a simple parallel sort where Morton index are used to compare particles.
We use two different approaches to sort the data.
In the next version of scalfmm the less efficient method should be deleted.

\subsection{Using QuickSort}
A first approach is to use a famous sorting algorithm.
We choose to use the quick sort algorithm because the distributed and the shared memory approaches are mostly similar.
Our implementation is based on the algorithm described in \cite{itpc03}.
The efficiency of this algorithm depends roughly of the pivot choice.
In fact, a wrong idea of the parallel quick sort is to think that each process first sort their particles using quick sort and then use a merge sort to share their results.
Instead, the nodes choose a common pivot and progress for one quick sort iteration together.
From that point all process has an array with a left part where all values are lower than the pivot and a right part where all values are upper or equal than the pivot.
Then, the nodes exchange data and some of them will work on the lower part and the other on the upper parts until there is one process for a part.
At this point, the process performs a shared memory quick sort.
To choose the pivot we tried to use an average of all the data hosted by the nodes:
\newline
\begin{algorithm}[H]
\linesnumbered
\SetLine
\KwResult{A Morton index as next iteration pivot}
\BlankLine
myFirstIndex $\leftarrow$ particles$[0]$.index\;
allFirstIndexes = MortonIndex$[nbprocs]$\;
allGather(myFirstIndex, allFirstIndexes)\;
pivot $\leftarrow$ Sum(allFirstIndexes(:) / nbprocs)\;
\BlankLine
\caption{Choosing the QS pivot}
\end{algorithm}
\newline
A bug was made when at the beginning, we did an average by summing all the values first and dividing after. But the Morton index may be extremly high, so we need to to divide all the value before performing the sum.

%%%%%%%%%%%%%%%%%%%%%%%%%%%%%%%%%%%%%%%%%%%%%%%%%%%%%%%%%%%%%%%%%%%%
%%%%%%%%%%%%%%%%%%%%%%%%%%%%%%%%%%%%%%%%%%%%%%%%%%%%%%%%%%%%%%%%%%%%
\subsection{Using a Sorting Network}
In \cite{ptttplwaefmm11}, a proposition has been made to sort the data using a sorting network.
We implemented a such sorting algorithm but the result were not extremly efficient.
Contrary to Quick sort, a sorting network is extremly stable and all the nodes performs similar work.
The quick sort is pivot dependant and some nodes may work much more than other.
But, the average case the quick sort enable higher efficiency.

%%%%%%%%%%%%%%%%%%%%%%%%%%%%%%%%%%%%%%%%%%%%%%%%%%%%%%%%%%%%%%%%%%%%
%%%%%%%%%%%%%%%%%%%%%%%%%%%%%%%%%%%%%%%%%%%%%%%%%%%%%%%%%%%%%%%%%%%%
\subsection{Using an intermediate Octree}
The second approach uses an octree to sort the particles in each process instead of a sorting algorithm.
The time complexity is equivalent but it needs more memory since it is not done in place.
After inserting the particles in the tree, we can iterate at the leaves level and access to the particles in an ordered way.
Then, the processes are doing a minimum and a maximum reduction to know the real Morton interval of the system.
By building the system interval in term of Morton index, the nodes cannot know the data scattering.
Finally, the processes split the interval in a uniform manner and exchange data with $P^{2}$ communication in the worst case.
\newline
\newline
In both approaches the data may not be balanced at the end.
In fact, the first method is pivot dependent and the second consider that the data are uniformly distributed.
That is the reason why we need to balance the data among nodes.

%%%%%%%%%%%%%%%%%%%%%%%%%%%%%%%%%%%%%%%%%%%%%%%%%%%%%%%%%%%%%%%%%%%%
%%%%%%%%%%%%%%%%%%%%%%%%%%%%%%%%%%%%%%%%%%%%%%%%%%%%%%%%%%%%%%%%%%%%
\section{Balancing the leaves}
After sorting, each process has potentially several leaves.
If we have two processes $P_{i}$ and $P_{j}$ with $i < j$ the sort guarantees that all leaves from node i are inferior than the leaves on the node j in a Morton indexing way.
But the leaves are randomly distributed among the nodes and we need to balance them.
It is a simple reordoring of the data, but the data has to stayed sorted.

\begin{enumerate}
\item Each process informs other to tell how many leaves it holds.
\item Each process compute how many leaves it has to send or to receive from left or right.
\end{enumerate}
At the end of the algorithm our system is completely balanced with the same number of leaves on each process.

\begin{figure}[h!]
\begin{center}
\includegraphics[width=15cm, height=15cm, keepaspectratio=true]{Balance.png}
\caption{Balancing Example}
\end{center}
\end{figure}

A process has to send data to the left if its current left limit is upper than its objective limit.
Same in the other side, and we can reverse the calculs to know if a process has to received data.

%%%%%%%%%%%%%%%%%%%%%%%%%%%%%%%%%%%%%%%%%%%%%%%%%%%%%%%%%%%%%%%%%%%%
%%%%%%%%%%%%%%%%%%%%%%%%%%%%%%%%%%%%%%%%%%%%%%%%%%%%%%%%%%%%%%%%%%%%
\chapter{Simple operators: P2M, M2M, L2L}
We present the different FMM operators in two separated parts depending on their parallel complexity.
In this first part, we present the three simplest operators P2M, M2M and L2L.
Their simplicity is explained by the possible prediction to know which node hosts a cell and how to organize the communication.

\section{P2M}
The P2M still unchanged from the sequential approach to the distributed memory algorithm.
In fact, in the sequential model we compute a P2M between all particles of a leaf and this leaf which is also a cell.
Although, a leaf and the particles it hosts belong to only one node so doing the P2M operator do not require any information from another node.
From that point, using the shared memory operator makes sense.

%%%%%%%%%%%%%%%%%%%%%%%%%%%%%%%%%%%%%%%%%%%%%%%%%%%%%%%%%%%%%%%%%%%%
%%%%%%%%%%%%%%%%%%%%%%%%%%%%%%%%%%%%%%%%%%%%%%%%%%%%%%%%%%%%%%%%%%%%
\section{M2M}
During the upward pass information moves from a level to the upper one.
The problem in a distributed memory model is that one cell can exist in several trees i.e. in several nodes.
Because the M2M operator computes the relation between a cell and its child, the nodes which have a cell in common need to share information.
Moreover, we have to decide which process will be responsible of the computation if the cell is present on more than one node.
We have decided that the node with the smallest rank has the responsibility to compute the M2M and propagate the value for the future operations.
Despite the fact that others processes are not computing this cell, they have to send the child of this shared cell to the responsible node.
We can establish some rules and some properties of the communication during this operation.
In fact, at each iteration a process never needs to send more than 7 cells, also a process never needs to receive more than 7 cells.
The shared cells are always at extremities and one process cannot be designed to be the responsible of more than one shared cell at a level.

\begin{figure}[h!]
\begin{center}
\includegraphics[width=14cm, height=7cm, keepaspectratio=true]{ruleillu.jpg}
\caption{Potential Conflicts}
\end{center}
\end{figure}

\begin{algorithm}[H]
\restylealgo{boxed}
\linesnumbered
\SetLine
\KwData{none}
\KwResult{none}
\BlankLine
\For{idxLevel $\leftarrow$ $Height - 2$ \KwTo 1}{
        \ForAll{Cell c at level idxLevel}{
                M2M(c, c.child)\;
        }
}
\BlankLine
\caption{Traditional M2M}
\end{algorithm}
\begin{algorithm}[H]
\restylealgo{boxed}
\linesnumbered
\SetLine
\KwData{none}
\KwResult{none}
\BlankLine
\For{idxLevel $\leftarrow$ $Height - 2$ \KwTo 1}{
        \uIf{$cells[0]$ not in my working interval}{
                isend($cells[0].child$)\;
                hasSend $\leftarrow$ true\;
        }
        \uIf{$cells[end]$ in another working interval}{
                irecv(recvBuffer)\;
                hasRecv $\leftarrow$ true\;
        }
        \ForAll{Cell c at level idxLevel in working interval}{
                M2M(c, c.child)\;
        }
        \emph{Wait send and recv if needed}\;
        \uIf{hasRecv is true}{
                M2M($cells[end]$, recvBuffer)\;
        }
}
\BlankLine
\caption{Distributed M2M}
\end{algorithm}
%%%%%%%%%%%%%%%%%%%%%%%%%%%%%%%%%%%%%%%%%%%%%%%%%%%%%%%%%%%%%%%%%%%%
%%%%%%%%%%%%%%%%%%%%%%%%%%%%%%%%%%%%%%%%%%%%%%%%%%%%%%%%%%%%%%%%%%%%
\section{L2L}
The L2L operator is very similar to the M2M.
It is just the contrary, a result hosted by only one node needs to be shared with every others nodes that are responsible of at least one child of this node.
\BlankLine
\begin{algorithm}[H]
\restylealgo{boxed}
\linesnumbered
\SetLine
\KwData{none}
\KwResult{none}
\BlankLine
\For{idxLevel $\leftarrow$ 2 \KwTo $Height - 2$ }{
        \uIf{$cells[0]$ not in my working interval}{
                irecv($cells[0]$)\;
                hasRecv $\leftarrow$ true\;
        }
        \uIf{$cells[end]$ in another working interval}{
                isend($cells[end]$)\;
                hasSend $\leftarrow$ true\;
        }
        \ForAll{Cell c at level idxLevel in working interval}{
                M2M(c, c.child)\;
        }
        \emph{Wait send and recv if needed}\;
        \uIf{hasRecv is true}{
                M2M($cells[0]$, $cells[0].child$)\;
        }
}
\BlankLine
\caption{Distributed L2L}
\end{algorithm}
%%%%%%%%%%%%%%%%%%%%%%%%%%%%%%%%%%%%%%%%%%%%%%%%%%%%%%%%%%%%%%%%%%%%
%%%%%%%%%%%%%%%%%%%%%%%%%%%%%%%%%%%%%%%%%%%%%%%%%%%%%%%%%%%%%%%%%%%%
\chapter{Complex operators: P2P, M2L}
These two operators are more complex than the ones presented in the previous chapter.
In fact, it is very difficult to predict the communication between nodes.
Each step requires pre-processing to know what are the potential communications and a gather to inform other about the needs.
\section{P2P}
To compute the P2P a leaf need to know all its direct neighbors.
Even if the Morton indexing maximizes the locality, the neighbors of a leaf can be on any node.
Also, the tree used in our library is an indirection tree.
It means that only the leaves that contain particles are created.
That is the reason why when we know that a leaf needs another one on a different node, this other node may not realize this relation if this neighbor leaf do not exist on its own tree.
At the contrary, if this neighbor leaf exists then the node wills require the first leaf to compute the P2P too.
In our current version we are first processing each potential needs to know the communication we should need.
Then the nodes do an all gather to inform each other how many communication they are going to send.
Finally they send and receive data in an asynchronous way and cover it by the P2P they can do.
\BlankLine
\begin{algorithm}[H]
\restylealgo{boxed}
\linesnumbered
\SetLine
\KwData{none}
\KwResult{none}
\BlankLine
\ForAll{Leaf lf}{
        neighborsIndexes $\leftarrow$ $lf.potentialNeighbors()$\;
        \ForAll{index in neighborsIndexes}{
                \uIf{index belong to another proc}{
                        isend(lf)\;
                        \emph{Mark lf as a leaf that is linked to another proc}\;
                }
        }
}
\emph{all gather how many particles to send to who}\;
\emph{prepare the buffer to receive data}\;
\ForAll{Leaf lf}{
        \uIf{lf is not linked to another proc}{
                neighbors $\leftarrow$ $tree.getNeighbors(lf)$\;
                P2P(lf, neighbors)\;
        }
}
\While{We do not have receive/send everything}{
	\emph{Wait some send and recv}\;
	\emph{Put received particles in a fake tree}\;
}
\ForAll{Leaf lf}{
	\uIf{lf is linked to another proc}{
	        neighbors $\leftarrow$ $tree.getNeighbors(lf)$\;
	        otherNeighbors $\leftarrow$ $fakeTree.getNeighbors(lf)$\;
	        P2P(lf, neighbors + otherNeighbors)\;
	}
}
\BlankLine
\caption{Distributed P2P}
\end{algorithm}
%%%%%%%%%%%%%%%%%%%%%%%%%%%%%%%%%%%%%%%%%%%%%%%%%%%%%%%%%%%%%%%%%%%%
%%%%%%%%%%%%%%%%%%%%%%%%%%%%%%%%%%%%%%%%%%%%%%%%%%%%%%%%%%%%%%%%%%%%
\section{M2L}
The M2L operator is relatively similar to the P2P.
Hence P2P is done at the leaves level, M2L is done on several levels from Height - 2 to 2.
At each level, a node needs to have access to all the distant neighbors of the cells it is the proprietary and those ones can be hosted by any other node.
Anyway, each node can compute a part of the M2L with the data it has.
The algorithm can be viewed as several tasks:
\begin{enumerate}
\item Compute to know what data has to be sent
\item All gather to know what data has to be received
\item Do all the computation we can without the data from other nodes
\item Wait $send/receive$
\item Compute M2L with the data we received
\end{enumerate}
\BlankLine
\begin{algorithm}[H]
\restylealgo{boxed}
\linesnumbered
\SetLine
\KwData{none}
\KwResult{none}
\BlankLine
\ForAll{Level idxLeve from 2 to Height - 2}{
        \ForAll{Cell c at level idxLevel}{
                neighborsIndexes $\leftarrow$ $c.potentialDistantNeighbors()$\;
                \ForAll{index in neighborsIndexes}{
                        \uIf{index belong to another proc}{
                                isend(c)\;
                                \emph{Mark c as a cell that is linked to another proc}\;
                        }
                }
        }
}
\emph{Normal M2L}\;
\emph{Wait send and recv if needed}\;
\ForAll{Cell c received}{
        $lightOctree.insert( c )$\;
}
\ForAll{Level idxLeve from 2 to Height - 1}{
        \ForAll{Cell c at level idxLevel that are marked}{
                neighborsIndexes $\leftarrow$ $c.potentialDistantNeighbors()$\;
                neighbors $\leftarrow$ lightOctree.get(neighborsIndexes)\;
                M2L( c, neighbors)\;
        }
}
\BlankLine
\caption{Distributed M2L}
\end{algorithm}
%%%%%%%%%%%%%%%%%%%%%%%%%%%%%%%%%%%%%%%%%%%%%%%%%%%%%%%%%%%%%%%%%%%%
%%%%%%%%%%%%%%%%%%%%%%%%%%%%%%%%%%%%%%%%%%%%%%%%%%%%%%%%%%%%%%%%%%%%
\begin{thebibliography}{9}
\bibitem{itpc03}
   Ananth Grama, George Karypis, Vipin Kumar, Anshul Gupta,
   \emph{Introduction to Parallel Computing}.
   Addison Wesley, Massachusetts,
   2nd Edition,
   2003.
\bibitem{ptttplwaefmm11}
   I. Kabadshow, H. Dachsel,
   \emph{Passing The Three Trillion Particle Limit With An Error-Controlled Fast Multipole Method}.
   2011.
\end{thebibliography}
\end{document}


