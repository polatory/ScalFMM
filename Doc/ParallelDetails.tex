\documentclass[10pt,letterpaper,titlepage]{report}
\usepackage{algorithm2e}
\usepackage{hyperref}
\usepackage{listings}

% latex test.tex ; dvipdf test.dvi
%%%%%%%%%%%%%%%%%%%%%%%%%%%%%%%%%%%%%%%%%%%%%%%%%%%%%%%%%%%%%%%%%%%%
%%%%%%%%%%%%%%%%%%%%%%%%%%%%%%%%%%%%%%%%%%%%%%%%%%%%%%%%%%%%%%%%%%%%

\author{Berenger Bramas}
\title{ScalFmm - Parallel Algorithms (Draft)}
\date{August, 2011}

\lstset{language=c++}

\setcounter{secnumdepth}{-1}

\begin{document}
\maketitle{}
\newpage

\tableofcontents
\newpage

%%%%%%%%%%%%%%%%%%%%%%%%%%%%%%%%%%%%%%%%%%%%%%%%%%%%%%%%%%%%%%%%%%%%
%%%%%%%%%%%%%%%%%%%%%%%%%%%%%%%%%%%%%%%%%%%%%%%%%%%%%%%%%%%%%%%%%%%%

\section{Introduction}
In this document we are going to introduce the principles and the algorithms used in our library to run in a distributed environement with MPI.
The algorithms in this document may not be up to date compare to those used in the code.

%%%%%%%%%%%%%%%%%%%%%%%%%%%%%%%%%%%%%%%%%%%%%%%%%%%%%%%%%%%%%%%%%%%%
%%%%%%%%%%%%%%%%%%%%%%%%%%%%%%%%%%%%%%%%%%%%%%%%%%%%%%%%%%%%%%%%%%%%

\chapter{Load a data file in Parallel}
\section{Description}
The main motivation to create a distributed version of the FMM is to run large simulation.
These simulations contains more particles than the memory of one computer.
That is the reason why it is not reasonable to ask a master process to load an entire file and to dispatch the data to other processes.

The solution we use can be view as a two steps process.
First, each process loads a part of the file and will have several particles.
Each process can compute the Morton index for the particles he has.
The Morton index of particle depends of the system but also of the tree height.
The second step is a parallel sort based on the Morton index with a balancing operation at the end.

%%%%%%%%%%%%%%%%%%%%%%%%%%%%%%%%%%%%%%%%%%%%%%%%%%%%%%%%%%%%%%%%%%%%
%%%%%%%%%%%%%%%%%%%%%%%%%%%%%%%%%%%%%%%%%%%%%%%%%%%%%%%%%%%%%%%%%%%%

\section{Load a file}
To split a file with all the processes we are using the MPI IO.
The prerequiste is to have a binary file to make the things easier.

\begin{lstlisting}
// From FMpiFmaLoader
particles = new FReal[bufsize];
MPI_File_set_view(file, headDataOffSet + startPart * 4 * sizeof(FReal),
	MPI_FLOAT, MPI_FLOAT, const_cast<char*>("native"), MPI_INFO_NULL);
MPI_File_read(file, particles, bufsize, MPI_FLOAT, &status);
\end{lstlisting}

It is easy to build the particles from a array of real that represents their position and a physical value.

%%%%%%%%%%%%%%%%%%%%%%%%%%%%%%%%%%%%%%%%%%%%%%%%%%%%%%%%%%%%%%%%%%%%
%%%%%%%%%%%%%%%%%%%%%%%%%%%%%%%%%%%%%%%%%%%%%%%%%%%%%%%%%%%%%%%%%%%%

\section{Sorting the particles}
\subsection{Using QuickSort}
A first approach is to use the quick sort algorithm.
Our implementation has been taken from \cite{itpc03}.
Efficiency of this algorithm dependands roughly of the choice of the pivot.
Whe are choosing our pivot as the average of the first Morton index of all array distributed on the processes.
This can be describe as:

\begin{algorithm}[H]
\SetLine
\KwData{none}
\KwResult{A Morton index as the pivot}
\BlankLine
firstIndexes $\leftarrow$ MortonIndex$[nbprocs]$\;
allGather(myFirstIndex, firstIndexes)\;
pivot $\leftarrow$ Sum(firstIndexes) / nbprocs\;
\BlankLine
\caption{Get a pivot for the next QS iteration}
\end{algorithm}

%%%%%%%%%%%%%%%%%%%%%%%%%%%%%%%%%%%%%%%%%%%%%%%%%%%%%%%%%%%%%%%%%%%%
%%%%%%%%%%%%%%%%%%%%%%%%%%%%%%%%%%%%%%%%%%%%%%%%%%%%%%%%%%%%%%%%%%%%

\subsection{Using an intermediate Octree}
The second approach is to use an octree to sort the particles in each process.
After inserting the particles in the tree, we can iterate at the leaves level and access to the particles in an ordered way.
Then, the processes are doing a minimum and a maximum reduction to know the real Morton interval of the system.
Finally, the processes are exchanging data with P² communication.

The result is, as with the Quick Sort, each process has some particles but this not balanced between proceeses.
In fact, at this time, the data has been cut using the entire Morton interval.

%%%%%%%%%%%%%%%%%%%%%%%%%%%%%%%%%%%%%%%%%%%%%%%%%%%%%%%%%%%%%%%%%%%%
%%%%%%%%%%%%%%%%%%%%%%%%%%%%%%%%%%%%%%%%%%%%%%%%%%%%%%%%%%%%%%%%%%%%

\section{Balancing the leaves}
After sorting, each process has leaves.
We know that for two processes Pi and Pj with i < j, all leaves hosted by Pi have their Morton indexes inferior of those hosted by Pj.
Using this information we are doing a two passes algorithm:
\begin{enumerate}
\item Each process has to know the number of leaves it has on its left and on its right
\item Each process compute to know if it has to send or receive something
\end{enumerate}
The result is that every processes have the same number of leaves.

%%%%%%%%%%%%%%%%%%%%%%%%%%%%%%%%%%%%%%%%%%%%%%%%%%%%%%%%%%%%%%%%%%%%
%%%%%%%%%%%%%%%%%%%%%%%%%%%%%%%%%%%%%%%%%%%%%%%%%%%%%%%%%%%%%%%%%%%%

\chapter{Simple operators: P2M, M2M, L2L}
This tree operators are more easy to understand.
They need only small message to be exchanged and moreover it is very easy to know which process to communicate with.

\section{P2M}
The P2M is unchanged from the sequential or shared memory model to the distributed memory one.
In fact, a leaf belong to only one process so doing the P2M operator do not require any information from other process.

%%%%%%%%%%%%%%%%%%%%%%%%%%%%%%%%%%%%%%%%%%%%%%%%%%%%%%%%%%%%%%%%%%%%
%%%%%%%%%%%%%%%%%%%%%%%%%%%%%%%%%%%%%%%%%%%%%%%%%%%%%%%%%%%%%%%%%%%%

\section{M2M}
During the upward pass information moves from one level to another.
The problem with the distributed memory is that one cell can exist in several trees ie in several processes.
We have decided that the rank with the smallest rank will have the responsability to compute the M2M and propagate the value for the futur operations.
Others processes have to send their child of this shared cell.
At each iteration a process never needs to send more than 8-1 cells, also process never needs to receive more than 8-1 cells.
In fact, the shared cells are always at the extrimity and one process cannot be designed to be the responsible of more than one shared cell at a level.

\begin{algorithm}[H]
\SetLine
\KwData{none}
\KwResult{none}
\BlankLine
\For{idxLevel $\leftarrow$ $Height - 2$ \KwTo 1}{
	\ForAll{Cell c at level idxLevel}{
		M2M(c, c.child)\;
	}
}
\BlankLine
\caption{Traditional M2M}
\end{algorithm}


\begin{algorithm}[H]
\SetLine
\KwData{none}
\KwResult{none}
\BlankLine
\For{idxLevel $\leftarrow$ $Height - 2$ \KwTo 1}{
	\uIf{$cells[0]$ not in my working interval}{
		isend($cells[0].child$)\;
		hasSend $\leftarrow$ true\;
	}
	\uIf{$cells[end]$ in another working interval}{
		irecv(recvBuffer)\;
		hasRecv $\leftarrow$ true\;
	}
	\ForAll{Cell c at level idxLevel in working interval}{
		M2M(c, c.child)\;
	}
	\emph{Wait send and recv if needed}\;
	\uIf{hasRecv is true}{
		M2M($cells[end]$, recvBuffer)\;
	}
}
\BlankLine
\caption{Distributed M2M}
\end{algorithm}

%%%%%%%%%%%%%%%%%%%%%%%%%%%%%%%%%%%%%%%%%%%%%%%%%%%%%%%%%%%%%%%%%%%%
%%%%%%%%%%%%%%%%%%%%%%%%%%%%%%%%%%%%%%%%%%%%%%%%%%%%%%%%%%%%%%%%%%%%

\section{L2L}
The L2L operator is very close to M2M.
There are shared cells and a process may need these cell to compute L2L on its own cells at a lower level.

\begin{algorithm}[H]
\SetLine
\KwData{none}
\KwResult{none}
\BlankLine
\For{idxLevel $\leftarrow$ 2 \KwTo $Height - 2$ }{
	\uIf{$cells[0]$ not in my working interval}{
		irecv($cells[0]$)\;
		hasRecv $\leftarrow$ true\;
	}
	\uIf{$cells[end]$ in another working interval}{
		isend($cells[end]$)\;
		hasSend $\leftarrow$ true\;
	}
	\ForAll{Cell c at level idxLevel in working interval}{
		M2M(c, c.child)\;
	}
	\emph{Wait send and recv if needed}\;
	\uIf{hasRecv is true}{
		M2M($cells[0]$, $cells[0].child$)\;
	}
}
\BlankLine
\caption{Distributed L2L}
\end{algorithm}

%%%%%%%%%%%%%%%%%%%%%%%%%%%%%%%%%%%%%%%%%%%%%%%%%%%%%%%%%%%%%%%%%%%%
%%%%%%%%%%%%%%%%%%%%%%%%%%%%%%%%%%%%%%%%%%%%%%%%%%%%%%%%%%%%%%%%%%%%

\chapter{Complexe operators: P2P, M2L}
This two operators are more complexe than the ones presented in the previous chapter.
In fact, they require a gather and a preprocess task that take time.

\section{P2P}
To process the P2P we need the neighbors of all our leaves.
But this neighbors can be potentialy hosted by any other process because of the Morton indexing.
Also, our tree is an indirection tree, so a leaf may not exist because there is particles at this place.
From that description of the problem we have implemented the fallowing algorithm.

\begin{algorithm}[H]
\SetLine
\KwData{none}
\KwResult{none}
\BlankLine

\ForAll{Leaf lf}{
	neighborsIndexes $\leftarrow$ $lf.potentialNeighbors()$\;
	\ForAll{index in neighborsIndexes}{
		\uIf{index belong to another proc}{
			isend(lf)\;
			\emph{Mark lf as a leaf that is linked to another proc}\;
		}
	}
}
\emph{all gather how many particles to send to who}\;
\emph{prepare the buffer to receive data}\;
\ForAll{Leaf lf}{
	\uIf{lf is not linked to another proc}{
		neighbors $\leftarrow$ $tree.getNeighbors(lf)$\;
		P2P(lf, neighbors)\;
	}
}
\emph{Wait send and recv if needed}\;
\emph{Put received particles in a fake tree}\;
\ForAll{Leaf lf}{
	\uIf{lf is linked to another proc}{
		neighbors $\leftarrow$ $tree.getNeighbors(lf)$\;
		otherNeighbors $\leftarrow$ $fakeTree.getNeighbors(lf)$\;
		P2P(lf, neighbors + otherNeighbors)\;
	}
}

\BlankLine
\caption{Distributed P2P}
\end{algorithm}

\section{M2L}
The M2L operator can be viewed as an P2P but done at each level of the tree.
So, at each level we need to have access to neighbors and those ones can be hosted by any processes.



%%%%%%%%%%%%%%%%%%%%%%%%%%%%%%%%%%%%%%%%%%%%%%%%%%%%%%%%%%%%%%%%%%%%
%%%%%%%%%%%%%%%%%%%%%%%%%%%%%%%%%%%%%%%%%%%%%%%%%%%%%%%%%%%%%%%%%%%%

\begin{thebibliography}{9}

\bibitem{itpc03}
   Ananth Grama, George Karypis, Vipin Kumar, Anshul Gupta,
   \emph{Introduction to Parallel Computing}.
   Addison Wesley, Massachusetts,
   2nd Edition,
   2003.

\end{thebibliography}

\end{document}
