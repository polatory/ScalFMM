
\section{Splitting the outer product in the interpolation polynomial}
\label{sec:splitouter}

We introduce the notation of the interpolation polynomial based on the
notation from \cite{mason2003chebyshev}, the interpolation of a function
$p(x)$ reads as
\begin{equation}
  \label{eq:masondef}
  p(x) \sim \sum_{n=0}^{\ell-1} \gamma_n c_n T_n(x)
\end{equation}
with $\gamma_0=1$ and $\gamma_n=2$ for $n\ge1$ and with
\begin{equation}
  \label{eq:chebcoeff}
  c_n = \frac{1}{\ell} \sum_{k=1}^\ell p(\bar x_n) T_k(\bar x_n)
\end{equation}
After inserting Eqn.~(\ref{eq:chebcoeff}) into Eqn.~(\ref{eq:masondef}) we end
up with
\begin{equation}
  \label{eq:outprod}
  p(x) \sim \frac{1}{\ell} \sum_{n=1}^\ell p(\bar x_n) \sum_{k=0}^{\ell-1}
  \gamma_k T_k(\bar x_n) T_k(x)
\end{equation}
In the discrete case, i.e., we have $N$ scattered points $\{x_i\}_{i=1}^N
\subset [-1,1]$, we write Eqn.~\eqref{eq:outprod} in matrix notation as
\begin{equation}
  \label{eq:interpolmatnot}
  \Mat{p} \sim \Mat{T}_x \Mat{\bar T}_x^\top \Mat{\bar p}
\end{equation}
with the matrices $\Mat{T}_x \in \mathbb{R}^{N\times\ell}$ with
$(\Mat{T}_x)_{ik} = T_{k-1}(x_i)$ and $\Mat{\bar T}_x \in
\mathbb{R}^{\ell\times\ell}$ with $(\Mat{\bar T}_x)_{nk} =
\nicefrac{\gamma_{k-1}}{\ell} \, T_{k-1}(\bar x_n)$. Using this notation the
interpolated kernel function $K(x,y)$ reads as
\begin{equation}
  \label{eq:kinterpolmatnot}
  \Mat{K} \sim \Mat{T}_x \Mat{\bar T}_x^\top \Mat{\bar K} (\Mat{T}_y
  \Mat{\bar T}_y^\top)^\top
\end{equation}
With the low-rank representation $\Mat{\bar K} \sim \Mat{UV}^T$ we compute
$\Mat{\bar U} = \Mat{\bar T}_x^\top \Mat{U}$ and $\Mat{\bar V} = \Mat{\bar
  T}_y^\top \Mat{V}$ and write Eqn.~\eqref{eq:kinterpolmatnot} as
\begin{equation}
  \label{eq:kinterpolmatnot1}
  \Mat{K} \sim \Mat{T}_x \Mat{\bar U} \Mat{\bar V}^\top \Mat{T}_y^\top
\end{equation}

\paragraph{Can we still use symmetries?} Recall the approach we proposed in
Sec.~\ref{sec:m2l}. It exploits the symmetries in the arrangement of the
far-field interactions. As we see from Eqn.~\eqref{eq:permut} in that case we
need permutation matrices $\Mat{P}_t$ for the $t$-th far-field interaction and
Eqn.~\eqref{eq:kinterpolmatnot} becomes
\begin{equation}
  \label{eq:symsplit}
  \Mat{K}_t \sim \Mat{T}_x \Mat{\bar T}_x^\top (\Mat{P}_t \Mat{\bar K}_{p(t)}
  \Mat{P}_t^\top) (\Mat{T}_y \Mat{\bar T}_y^\top)^\top.
\end{equation}
With $\Mat{\bar K}_{p(t)} \sim \Mat{U}_{p(t)} \Mat{V}_{p(t)}^\top$ the
matrices in the outer product in Eqn.~\eqref{eq:kinterpolmatnot1} become
$\Mat{\bar U}_t = \Mat{\bar T}_x^\top \Mat{P}_t \Mat{U}_{p(t)}$ and $\Mat{\bar
  V}_t = \Mat{\bar T}_y^\top \Mat{P}_t \Mat{V}_{p(t)}$. Thus, it is not
possible anymore to use the fact that due to symmetries we can express all
$316$ far-field interactions by permutations of $16$ only.





%%% Local Variables: 
%%% mode: latex
%%% TeX-master: "main"
%%% End: 
